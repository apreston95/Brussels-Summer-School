%\documentclass[]{beamer}
%\documentclass[handout]{beamer} 
\documentclass[aspectratio=169,handout]{beamer}
%\documentclass[aspectratio=169]{beamer}

%%%%%%%%%%%%%%%%%%%%%%%%%%%%%%%%%%%%%%%%%%%%%%%%%%%%%%%%%%%%%%%%%%%%%%%%%%%%%%%%%%%%%%%%%%%%%%%%%%%%%%%%%%%%%%%%%%%%%%%%%%%%%%%%%%%%%%%%%%%%%%%%%%%%%%%%%%%%%%%%%%%%%%%%%%%%%%%%%%%%%%%%%%%%%%%%%%%%%%%%%%%%%%%%%%%%%%%%%%%%%%%%%%%%%%%%%%%%%%%%%%%%%%%%%%%%

\usepackage[french]{babel}
\usepackage[latin1]{inputenc}
\usepackage{hyperref,eurosym,colortbl}
\hypersetup{colorlinks=true,linkcolor=gray,urlcolor=gray,pdfpagemode=UseNone}
\usepackage{graphicx,tikz,ifthen} 
\usepackage{pgfplots}
\usepackage{multimedia}
\usepackage{pgfnodes,pgfarrows,pgfheaps}

\usetikzlibrary{decorations.markings}
\usetikzlibrary{patterns}



\setbeamertemplate{caption}[numbered]
\usetheme[]{default}
\setbeamertemplate{footline}[frame number]

\setbeamercolor{title}{fg=gray}
\setbeamercolor{frametitle}{fg=gray}
\setbeamercolor{caption}{fg=gray}
\setbeamercolor{figurenumber}{fg=gray}
\setbeamercolor{caption name}{fg=gray}
\setbeamercolor{item}{fg=gray}
\setbeamertemplate{itemize subitem}{ \textcolor{gray}{\boldmath $\times$} }

\setbeamercolor{block title}{use=structure,fg=gray,bg=white}
\setbeamercolor{block body}{use=structure,fg=black,bg=white}

\setbeamertemplate{theorems}[numbered]
\beamertemplatenavigationsymbolsempty
\beamertemplatetransparentcovered

\definecolor{grey}{gray}{0.5}
\definecolor{dockergray}{rgb}{0.11,0.56,0.98}
\definecolor{onegray}{rgb}{0,0,0.75}
\definecolor{freegray}{rgb}{0.25,0.41,0.88}
\definecolor{mygray}{rgb}{0,0.2,0.4}

\definecolor{stabilo}{rgb}{.8242,.8242,.8242}
\definecolor{dockerblue}{rgb}{0.11,0.56,0.98}
\definecolor{oneblue}{rgb}{0,0,0.75}
\definecolor{freeblue}{rgb}{0.25,0.41,0.88}
\definecolor{myblue}{rgb}{0,0.2,0.4}

\usetikzlibrary{snakes}



\newtheorem{prop}{Proposition}


\newboolean{note}
%\setboolean{note}{true}
\setboolean{note}{false}



\begin{document}

\begin{frame}{Understanding ``$V$ functions''}
\begin{itemize}
\item Here we take simple static general equilibrium models and derive the ``$V$ function''
\item We show that although there is an ``externality'', competitive equilibrium is efficient.
\end{itemize}
\end{frame}


\begin{frame}{Understanding ``$V$ functions''}{Model}
\begin{itemize}
\item Consider a simple consumption labor supply problem, where agent $i$ has preferences $U(c_{i},e_{i})$, and has the budget constraint 
$$
c_{i} = w e_{i} + \frac{\Pi}{N}
$$
 where $\frac{\Pi}{N}$ is the share of profits 
 \item There are $N$ agents and they are all the same
 \item We will focus on symmetric  allocations $e_{j} = e$ $\forall j$
\item There is a competitive firm with technology $F(N e)$
\item The consumption good is the num\'eraire
\end{itemize}
\end{frame}

\begin{frame}{Understanding ``$V$ functions''}{To do}
\begin{enumerate}
\item From firms behaviour, derive equilibrium wage as a function of $Ne$.
\item Use it to write household $i$ utility as a function $V(e_{i}, e)$, when the wage function is the competitive one. 
\item Find household $i$ FOC and the equation that defined Walrasian GE
\item Now take function $V$ and wage function and look for a planner symmetric solution, that would internalise the wage effect. 
\item Show that this solution is the same than the competitive one. 
\end{enumerate}
\end{frame}


\begin{frame}{Understanding ``$V$ functions''}{Competitive Equilibrium}

\begin{itemize}
\item Firm profit:
$$
\Pi = F(Ne) - w N e
$$

\item Firm profit maximisation leads to 
$$
F'(N e)=w \mbox{ or equivalently } w=w(N e)
$$

\item so that 
$$
\Pi(Ne) = F(Ne) - w(Ne) N e
$$

\end{itemize}
\end{frame}

\begin{frame}{Understanding ``$V$ functions''}{Competitive Equilibrium}


\begin{itemize}
\item In this setup, one can write household $i$ decision problem using the  following $V$ function: 
$$
V(e_{i}, e) = U\left(w(Ne) e_{i} + \frac{\Pi(Ne)}{N},e_{i}\right)
$$
\item The individual FOC is 
$$
w U_{1} + U_{2} = 0
$$
\end{itemize}
\end{frame}


\begin{frame}{Understanding ``$V$ functions''}{Planner Solution using the $V$ function}

\begin{itemize}
\item Let us now consider the social optimum
\item We have $V(e, e) = U\left(w(Ne) e + \frac{\Pi(Ne)}{N},e\right)$
\item Planner FOC is 
$$
V_{1}+V_{2}=0
$$
\item Note that
$$
\frac{\Pi(Ne)}{N} = \frac{F(Ne) - w(Ne)Ne}{N}
$$
and
\begin{align*}
\frac{\partial\frac{\Pi(Ne)}{N}}{\partial e} &= \frac{1}{N} \left[NF'(Ne) - w \frac{\partial Ne}{\partial e} - \frac{\partial w}{\partial e} Ne\right]\\
&= \left(F'(Ne) -w\right)  -\frac{\partial w}{\partial e} e
\end{align*}

\end{itemize}
\end{frame}


\begin{frame}{Understanding ``$V$ functions''}{Planner Solution using the $V$ function}

\begin{itemize}
\item For the Planner:
$$
V_{1} = w U_{1} + U_{2}
$$
and
$$
V_{2} = U_{1} \left[\frac{\partial w}{\partial e} e + \left(F'(Ne) -w\right) -\frac{\partial w}{\partial e} e\right]
$$
\item Because $F'(N e)=w$, this implies 
$$
V_{2} = 0
$$

\end{itemize}
\end{frame}


\begin{frame}{Understanding ``$V$ functions''}{Planner Solution using the $V$ function}

\begin{itemize}
\item Planner FOC is  therefore
$$
V_{1}= w U_{1} + U_{2}=0
$$
\item So that we are back to the competitive equilibrium condition. 
\item As we see, the actions of the others have an impact on one's payoff $V$, but nevertheless there are no inefficiencies at the Walrasian equilibrium. 
\end{itemize}
\end{frame}





\end{document}

 


